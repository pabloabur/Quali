\pretextualchapter{Agradecimentos}
Ao Prof. Dr. Andr�, pela excelente orienta��o e pelos ensinamentos
inestim�veis, tanto na vida pessoal quanto profissional.
Ao Prof. Dr. Renato, pelo apoio, incentivo e ideias durante
todo o mestrado.

� minha m�e, meu pai, meu irm�o e minhas av�s pelo apoio
incondicional e por sempre me inspirar.

Aos colegas do LEB Felipe, Fernando, Roberto e Ana pelo apoio
e amizade.

Aos meus amigos Matheus Ferreira, Lucas Severo e Daniel Cunha que,
apesar da dist�ncia, nunca deixam de estar presentes. Aos meus amigos
Felipe Marangoni e Rodrigo Siqueira, pelos conselhos e por ter tornado
minha estadia em S�o Paulo muito mais agrad�vel. Um agradecimento especial
� Joice Vieira, pelo apoio, carinho e pela paci�ncia.

� Superintend�ncia de Tecnologia da Informa��o da Universidade de S�o Paulo,
pelo aux�lio com os recursos de HPC, e � Ana Maria, por me ajudar com
a formata��o final desse trabalho.

O presente trabalho foi realizado com apoio do CNPq, Conselho Nacional de
Desenvolvimento Cient�fico e Tecnol�gico - Brasil (132587/2018-1).
