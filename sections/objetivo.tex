\chapter{Objetivos}
\section{Avalia��o do Circuito de Inibi��o Recorrente}
O primeiro objetivo deste trabalho � avaliar como as CRs afetam o
comportamento dos MNs envolvidos na inibi��o recorrente, analisando se os
resultados s�o consistentes com dados fisiol�gicos. Como a estima��o de 
par�metros j� foi realizada em \citeonline{cisi08}, o foco aqui ser�o os
comportamentos globais desse circuito.

Esses aspectos ser�o analisados comparando os resultados obtidos com
simula��es com a parametriza��o antiga e nova. Isso deve trazer informa��es
sobre qual parametriza��o � mais adequada em cada situa��o e porqu�.

\section{Melhora de Performance}
Como mostrado anteriormente, o tempo de execu��o de uma simula��o se torna um 
fator limitante quando se deseja estudar sistemas mais realistas. Sendo assim,
outro objetivo desse trabalho � melhorar a performance computacional do
simulador em Python. Apesar deste ser significativamente mais lento que a
vers�o em Java, optou-se por trabalhar com esta porque, atualmente, ela est� em
constante manuten��o e desenvolvimento, al�m de ser mais facilmente estendida
para realizar simula��es mais completas.

Vale ressaltar que esse n�o � o foco desse trabalho, pois esse tipo de
desenvolvimento pode ser bastante complexo e demorado por si s�.

\section{Variabilidade de For�a}
Como objetivo final, deseja-se estudar o efeito das CRs na variabilidade de
for�a em tarefas de for�a em um pedal fixo e solto.

