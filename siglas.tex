%%% modifica a maneira de apresentar a lista de siglas
	\def\bflabel#1{{{\textsf{#1}}\hfill}}
	\renewenvironment{AC@deflist}[1]%
	      {\if AC@nolist%
		\else%
		    \raggedright\begin{list}{}%
		    {\settowidth{\labelwidth}{\textsf{#1}+8pt}%
		    \setlength{\leftmargin}{\labelwidth}%
		    \setlength{\itemsep}{0.5pt}%
		    \addtolength{\leftmargin}{\labelsep}%
		    \renewcommand{\makelabel}{\bflabel}}%
		 \fi}%
		{\if AC@nolist%
		  \else%
		    \end{list}%
		\fi}%

\pretextualchapter{Lista de Siglas}

\begin{acronym}[NARMAX]
    \acro{SNC}[SNC]{Sistema Nervoso Central}
    \acro{MN}[MN]{Motoneur�nio}
    \acro{IN}[IN]{Interneur�nio}
    \acro{ACh}[ACh]{Acetilcolina}
    \acro{PEPS}[PEPS]{Potencial Excitat�rio P�s Sin�ptico}
    \acro{FF}[FF]{Fast Fatiguing}
    \acro{FR}[FR]{Fast Fatigue Resistant}
    \acro{S}[S]{Slow}
    \acro{CR}[CR]{C�lula de Renshaw}
    \acro{PIPS}[PIPS]{Potencial Inibit�rio P�s Sin�ptico}
    \acro{AHP}[AHP]{Afterhyperpolarization}
    \acro{FxI}[FxI]{frequ�ncia de disparo \textit{versus} corrente injetada}
    \acro{$g_{max_{FF}}$}[$g_{max_{FF}}$]{Condut�ncia m�xima de um
            motoneur�nio sobre uma c�lula de Renshaw}
\end{acronym}
