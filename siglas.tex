%%% modifica a maneira de apresentar a lista de siglas
	\def\bflabel#1{{{\textsf{#1}}\hfill}}
	\renewenvironment{AC@deflist}[1]%
	      {\if AC@nolist%
		\else%
		    \raggedright\begin{list}{}%
		    {\settowidth{\labelwidth}{\textsf{#1}+8pt}%
		    \setlength{\leftmargin}{\labelwidth}%
		    \setlength{\itemsep}{0.5pt}%
		    \addtolength{\leftmargin}{\labelsep}%
		    \renewcommand{\makelabel}{\bflabel}}%
		 \fi}%
		{\if AC@nolist%
		  \else%
		    \end{list}%
		\fi}%

\pretextualchapter{Lista de Siglas}

\begin{acronym}[NARMAX]
    \acro{ACh}[ACh]{Acetilcolina}
    \acro{AHP}[AHP]{Hiperpolariza��o p�s-potencial de a��o (do ingl�s \textit{Afterhyperpolarization}}
    \acro{CR}[CR]{C�lula de Renshaw}
    \acro{CST}[CST]{Trem de disparos cumulativo (do ingl�s \textit{Cumulative Spike Train})}
    \acro{CVM}[CVM]{Contra��o volunt�ria m�xima}
    \acro{EMG}[EMG]{Eletromiografia}
    \acro{FxI}[FxI]{Frequ�ncia de disparo \textit{versus} corrente injetada}
    \acro{FF}[FF]{Fast fatiguing}
    \acro{FR}[FR]{Fast fatigue resistant}
    \acro{IN}[IN]{Interneur�nio}
    \acro{MN}[MN]{Motoneur�nio}
    \acro{PEPS}[PEPS]{Potencial excitat�rio p�s-sin�ptico}
    \acro{PIPS}[PIPS]{Potencial inibit�rio p�s-sin�ptico}
    \acro{S}[S]{Slow}
    \acro{SNC}[SNC]{Sistema nervoso central}
\end{acronym}
